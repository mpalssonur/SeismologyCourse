\section*{Problem 4}
Show that the principal stress axes always coincide with the principal strain axes for isotropic media. In other words, show that if $\mathbf{x}$ is an eigenvector of $\mathbf{e}$, then it is also an eigenvector of $\boldsymbol{\tau}$.

\subsection*{Solution}

We use the relationship
\begin{equation*}
    \boldsymbol{\tau} = 
    \lambda \text{tr}(\mathbf{e})\mathbf{I} +
    2\mu \mathbf{e}
\end{equation*}
and the definition of eigenvectors/eigenvalues of a matrix, $\mathbf{ex} = a \mathbf{x}$, using $a$ rather than the more traditional $\lambda$ to avoid confusion. Note that the trace of a matrix is a scalar.
\begin{align*}
    \boldsymbol{\tau}\mathbf{x} &= (\lambda \text{tr}(\mathbf{e})\mathbf{I} + 2\mu \mathbf{e}) \mathbf{x} \\
    &= \lambda \text{tr}(\mathbf{e})\mathbf{I} \mathbf{x} + 
    2\mu \mathbf{e}\mathbf{x} \\
    &= \lambda \text{tr}(\mathbf{e})\mathbf{x} + 
    2\mu a \mathbf{x} \\
    &= (\lambda \text{tr}(\mathbf{e}) + 
    2\mu a) \mathbf{x} \\
    &= b \mathbf{x}
\end{align*}
So $\mathbf{x}$ is an eigenvector of $\boldsymbol{\tau}$ with the eigenvalue of $b = (\lambda \text{tr}(\mathbf{e}) + 2\mu a)$.\\

From eqn. 2.30 in the book we can calculate that $\text{tr}(\boldsymbol{\tau}) = 3\lambda\text{tr}(\mathbf{e}) + 2 \mu \text{tr}(\mathbf{e})$ meaning that 
\begin{equation*}
    \text{tr}(\mathbf{e}) = \frac{\text{tr}(\boldsymbol{\tau})}{3\lambda + 2\mu}
\end{equation*}

giving us the eiginvalue of $\boldsymbol{\tau}$, $b$ corresponding to the eigenvector $\mathbf{x}$

\begin{equation*}
    b = \frac{\lambda}{3\lambda + 2\mu} \text{tr}(\boldsymbol{\tau}) + 2\mu a
\end{equation*}
